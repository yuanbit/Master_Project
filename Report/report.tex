\documentclass[12pt,english]{article}
\usepackage[a4paper,bindingoffset=0.2in,%
            left=1in,right=1in,top=1in,bottom=1in,%
            footskip=.25in]{geometry}
\usepackage{blindtext}

\usepackage[english]{babel}
\usepackage[utf8]{inputenc}
\usepackage{amsmath}
\usepackage[colorinlistoftodos]{todonotes}
\usepackage{graphicx}
\usepackage{amssymb}
\usepackage[backend=bibtex]{biblatex}
\addbibresource{mybib.bib}
\usepackage[hidelinks]{hyperref}


\title{Correlation between Professional Domain and Facial Features based on Face Clustering}

\author{Bithiah Yuan}

\date{\today}

\begin{document}

\maketitle

%\begin{abstract}
%Bitcoin as an electronic payment system prevents the double-spending problem using the Proof-of-Work confirmation protocol. However, security issues arise with fast payments using Bitcoin as merchants are required to exchange their goods in a short time. We examine double-spending attacks on fast-payments and show how the attacks can be executed. Moreover, we demonstrate how the countermeasures by Bitcoin developers are either ineffective or produce additional costs for the merchants. Ultimately, we present an easy and lightweight solution for the dection of double-spending attacks in fast payments.


%\end{abstract}

\section{Introduction}
\label{sec:introduction}

\quad 
A significant source of information and attributes can be derived from the human face by non-verbal communication \cite{joo}. As a result, facial features have been studied extensively in the social-science domain to predict success in reaching reputable leadership positions. In particular, studies have shown that certain facial features contribute to higher salaries and more prestigious employments for CEOs. In application, the relationship between facial characteristics and social attributes can provide an more powerful objective indicator for organizations to idenity and select effective leaders within their domain than broad facial cues such as attractiveness and competence. Results have shown that a human judge can identify business, military, and sports leaders from their faces with above-chance accuracy. However, these results are biased and do not imply the actual leadership qualities of a person \cite{olivola}.

Caused by behaviour experiments from human judgement, the the research of the social attributes and facial features in the social-sciences are limited in scalability, consistency, and generalization. For example, prior familiarity to the faces of the study and personal preferences can affect the results. Therefore, a growing number of social trait judgment studies have been extended and refined to computer vision and machine learning research due to the capability of using massive datasets and large-scale processing capacity \cite{joo}.

Through a computational framework, \cite{joo} examined the relationship between facial traits and the social construction of leadership by a trained model that can predict the outcomes of political elections based on the perceived social attributes of a person's appearance. The results indicate that similar methods can be used to predict behavior in a broad range of human social relations, such as mate selection, job placement,and political and commercial negotiations \cite{joo}.

Clustering analysis is an unsupervised learning technique that groups data points into clusters based on their similarities. It is useful in grouping a collection of unlabeled data with similar nature into clusters. \cite{shi} investigated clustering a large number of unlabeled face images into individual identities present in the data \cite{shi}. The workflow shown in Figure\ref{fig:face}. consists of obtaining face representations of a collection of unlabeled date by a deep neural network. The choice of clustering algorithm then groups the face images according to their identity.

Motivated by the researches in computational social trait judgment and \cite{shi}, the following paper aims to examine the correlation between a person's profession based on their facial features through clustering face images. The clustering problem consists of the face representation and similarity metric of the face images and the choice of clustering algorithm \cite{shi}. Due to the importance of the underlying face representation in face clustering, this paper further compares different open-source state-of-the-art feature extraction methods based on deep learning.\\

\begin{figure}[!tbp]
 \centering
    \includegraphics[width=\textwidth]{figures/otto_faceClustering_workflow.png}
    \caption{Face clustering workflow \cite{shi}}
	\label{fig:face}
\end{figure}

\section{Related Work}	

\subsection{Face Recognition}

\quad
Face recognition focuses on identifying or verifying the identity of subjects in images or videos \cite{trigueros}.\\

Face recognition systems are usually composed of the following 4 steps \cite{trigueros}: 

\begin{enumerate}
  \item Face Detection:\\ Detect the position of the faces in an image and returns the coordinates of a bounding box for each face as shown in Figure \ref{fig:detect}.
  \item Face Alignment: \\ Find a set of facial landmarks with the best affine transformation that fits a set of reference points located at fixed locations in the image as shown in Figure \ref{fig:landmark}.
  \item Face Representation: \\ Trasnform the pixel values of a face image into a discriminative feature vector.
  \item Face Matching: \\ Compute similarity scores from feature vectors.
\end{enumerate} 
 
\begin{figure}[!tbp]
  \centering
  \begin{minipage}[b]{0.49\textwidth}
    \includegraphics[width=\textwidth]{figures/face_detection.png}
    \caption{Face Detection \cite{trigueros}}
    \label{fig:detect}
  \end{minipage}
  \hfill
  \begin{minipage}[b]{0.49\textwidth}
    \includegraphics[width=\textwidth]{figures/landmark.png}
    \caption{Face Alignment \cite{trigueros}}
    \label{fig:landmark}
  \end{minipage}
\end{figure}

\subsection{Face Detection}

\subsubsection{Histograms of Oriented Gradients}

\quad
\cite{dalal}
The method is based on evaluating well-normalized local histograms of image gradient orientations in a dense grid. The basic idea is that local object appearance and shape can often be characterized rather well by the distribution of local intensity gradients or edge directions, even without precise knowledge of the corresponding gradient or edge positions. In practice this is implemented by dividing the image window into small spatial regions (“cells”), for each cell accumulating a local 1-D histogram of gradient directions or edge orientations over the pixels of the cell. The combined histogram entries form the representation. For better invariance to illumination, shadowing, it is also useful to contrast-normalize the local responses before using them. This can be done by accumulating a measure of local histogram “energy” over somewhat larger spatial regions (“blocks”) and using the results to normalize all of the cells in the block. We will refer to the normalized descriptor blocks as Histogram of Oriented Gradient (HOG) descriptors. 

\subsection{Face Representation}

\subsubsection{Convolutional Neural Networks}

\quad
Face representation is conceivably the most important component in the system \cite{trigueros}. However, challenges occur in real world (in-the-wild) images due to variations ranging from head poses and illumination conditions to aging and facial expressions. Recently, deep learning methods based on convolutional neural networks (CNNs) were able to achieve very high accuracy by learning robust features due to the availability of large-scale faces in-the-wild datasets on the web \cite{trigueros}. 

Residual networks (ResNets) is a popular network architecture for face recognition. ResNets introduces a shortcut connection to learn a residual mapping which contributes to information flow across layers and allows the training of much deeper architectures \cite{trigueros}.

A common approach to training CNN models for face recognition is use a classification approach, where each face image in the training set corresponds to a class. When recognizing a new face image, the classification layer is discarded and the features of the previous layer are used as face representations. The downsides of this approach is that it doesn't generalize well to new face faces and that the representation size per face is large and inefficient \cite{schroff}.

Another approach is to learn the features for face representation directly by optimizing the distance between pairs or triplets of faces, in which the distances measure the similarity between faces \cite{trigueros} \cite{schroff}. 

\subsubsection{Triplet Loss Function}

\quad
When learning the face features directly, the choice of loss function has a great influence on the accuracy. One of the most used metric is the triplet loss function. The goal of the loss is to separate the distance between two aligned matching (positive) face images and a non-matching aligned (negative) face image by a distance margin. The result is a feature vector $f(x)$, known as embeddings, from a face image $x$ to a compact Euclidean feature space in $ \mathbb{R}^{d}$. The distance of the embeddings will be small if the faces are identical and large if the faces are distinct \cite{schroff}.


More specifically, as shown in the example in Figure \ref{fig:bale}. the distance between an anchor face image, $x_{i}^{a}$ is minimized by the loss and will be closer to all other positive face images $x_{i}^{p}$ than the negative face images $x_{i}^{n}$ where the distance is maximized by the loss. For each triplet $i$, the following condition needs to be satisfied: $$\Vert f(x_{i}^{a}) - f(x_{i}^{p}) \Vert_{2}^{2} + \alpha < \Vert f(x_{i}^{a}) - f(x_{i}^{n}) \Vert_{2}^{2} $$
where $\alpha$ is a margin that from the positive and negative pairs \cite{trigueros}.

For $N$ possible triplets, the loss being minimized is: $$ L = \sum_{i}^{N} \Big[ \Vert f(x_{i}^{a}) - f(x_{i}^{p}) \Vert_{2}^{2} - \Vert f(x_{i}^{a}) - f(x_{i}^{n}) \Vert_{2}^{2} + \alpha\Big]_{+} $$ \cite{schroff}.

\begin{figure}[!tbp]
 \centering
    \includegraphics[width=\textwidth]{figures/triplet_loss_example.png}
    \caption{The loss of identical faces are minimized and the loss of distinct faces are maximized by the triplet loss function.}
	\label{fig:bale}
\end{figure}
 
\section{Face Representation Methods}

\subsection{FaceNet}

\quad

FaceNet is a method that uses a deep CNN and the triplet loss function to directly optimize the face embeddings. Between 100 to 200 million face images consisting of about 8 million different identities were used for training. The large dataset if labelled faces consist of various poses, illuminations, and other variations. \cite{schroff}.

\section{Experiment}

\subsection{FaceNet}

This page describes the training of a model using the VGGFace2 dataset and softmax loss. The dataset contains 3.31 million images of 9131 subjects (identities), with an average of 362.6 images for each subject. Images are downloaded from Google Image Search and have large variations in pose, age, illumination, ethnicity and profession (e.g. actors, athletes, politicians).
20180402-114759
0.9965
VGGFace2
Inception ResNet v1

\subsection{Dataset}


\subsection{Evaluation}


\section{Conclusion}

\nocite{*}

\printbibliography

\end{document}